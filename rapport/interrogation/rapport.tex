\documentclass[12pt,titlepage]{article}

\usepackage{float}
\usepackage[T1]{fontenc}
\usepackage[utf8]{inputenc}
\usepackage[french]{babel} 
\usepackage{amsmath}
\usepackage{amssymb}
\usepackage[top=1.5cm, bottom=1.5cm, left=1.5cm, right=1.5cm]{geometry}
\usepackage{graphicx}

\begin{document}

\begin{titlepage}
\newcommand{\HRule}{\rule{\linewidth}{0.5mm}}
\center
\textsc{\LARGE
Université de Montpellier
} \\[1cm]
\begin{figure}[h]
	\begin{minipage}[c]{.46\linewidth}
		\centering
		\includegraphics[width=1\textwidth]{img/fds.png}
	\end{minipage}
	\hfill%
	\begin{minipage}[c]{.46\linewidth}
		\centering
		\includegraphics[width=1\textwidth]{img/univ-montpellier.png}
	\end{minipage}
\end{figure}

\HRule \\[0.4cm]
{ \huge \bfseries Rapport de projet NoSQL \\ Partie I - évaluation des requêtes en étoile}
\HRule \\[1.5cm]
El Houiti Chakib \\
Kezzoul Massili
\\[1cm]
\today \\ [1cm]
\end{titlepage}

\section*{Introduction}

\subsection*{Objectifs du projet}

Quels sont, en bref, les objectifs du projet ?

\subsection*{Environnement de développement}

Utilisation de Maven, VSCode et d'un Makefile au lieu de l'environnement Eclipse. Pourquoi ?
Détailler l'environnement de développement (Github tout ça tout ça).

\subsection*{Structure du projet}

Expliquer brievement la structure (physique) du projet. (Détailler les fichiers, les dossiers, les répertoires, etc.)

\begin{description}
	\item[\textit{src/}] ...
	\item[\textit{target/}] ...
	\item[\textit{data/}] ...
	\begin{description}
		\item[Jeu de données] fichiers.nt
		\item[Requêtes] fichiers.queryset
	\end{description}
		\item[\textit{output/}] ...	
		\item[\textit{README.md}] ...
		\item[\textit{Makefile}] ...
		\item[\textit{pom.xml}] ...
\end{description}

\section{Modèlisations et implémentation}

\subsection{Chargement des données}

\subsubsection{Jeu de données}

\subsubsection{Dictionnaire}

\subsubsection{Indexes}

\subsection{Évaluation des requêtes}

\subsubsection{Structure}

\subsubsection{Lecture}

\subsubsection{Évaluation}

\paragraph{matchPattern}

\paragraph{Intersection de chaque pattern}

\subsection{Résultats}

\paragraph{Structure}

Fichier CSV en sortie du programme.

\paragraph{Validation}

Étape de validation des résultats.

\section{Conclusion}

\subsection{Utilisation du programme}

Parler des différents arguments de la ligne de commande ainsi que de leur fonctionnement.

\subsection{Perspectives}

Évaluer le temps d'exécution des requêtes.

\end{document}